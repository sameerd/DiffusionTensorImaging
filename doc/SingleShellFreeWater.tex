\documentclass[12pt]{article}
\usepackage{hyperref}
\usepackage{amsmath}

\newcommand{\vect}[1]{\mathbf{#1}}


\title{Single Shell Free Water Elimination Model Implemention}
\author{
  Sameer DCosta \\
    sameerdcosta@gmail.com
}
\date{\today}


\begin{document}
\maketitle

\section{Introduction}
This document details out an implementation of a Free Water Elimination model
for Diffusion Tensor MRI. We are mainly following the implementation in
\cite{Pasternak2009} with a few simplifications suggested in \cite{Pasternak2014}.

\section{Bi-tensor Model}
In this section we summarize the \textbf{Theory} section in
\cite{Pasternak2009}. 

\ 

\noindent
For each voxel we have several readings. $\vect{S}_0$ is the signal acquired
for the zero diffusion weighting from the $\vect{b0}$ image. $\vect{S}_k$ is
the signal from the diffusion weighted image (DWI) when the gradient
orientation $\vect{q}_k$ is applied. We calculate the attenuation
$[\hat{\vect{A}}]_k = \vect{S}_k / \vect{S}_0$. The attenuation values are
between 0 and 1. 

\ 

\noindent
For the \textbf{single compartment model} we assume that the attentuation
$[\hat{\vect{A}}]_k$ all comes from tissue $[\vect{A}_{\text{tissue}}]_k$. For a Diffusion Tensor $\vect{D}$ this gives us 
$$[\vect{A}_{\text{tissue}}(\vect{D})]_k = \exp(-b
\vect{q}_k^T\vect{D}\vect{q}_k).$$
The $b$ value in the equation is above is the b-value of our single shell. The
gradient $\vect{q}_k$ is a vector of length 3 and $\vect{D}$ is a 3x3 symmetric
matrix at each voxel. Usually we have a large number of $k$. The minimium
required is 6 as $\vect{D}$ has 6 parameters that need to be estimated at each
voxel.  However in practice we have somewhere between 30 and 60. 

\ 

\noindent
If we had water instead of tissue at a voxel then DTI model (above) simplifies. The matrix $\vect{D}$ becomes a scalar $d$ and we get
$$[\vect{A}_{\text{water}}]_k = \exp(-bd),$$ where $d = 3 \cdot 10^{-3}$
$\text{mm}^2/\text{s}$ for water at $37^\circ$C.




\section{Note}
The definition of $p^i$ and $q^i$ is a generalization of the ones given in the
discretization of the Beltrami flow given in \cite{Kimmel}.

\begin{thebibliography}{9}
        \bibitem[Pasternak2009]{Pasternak2009} Pasternak, O. , Sochen, N. ,
                Gur, Y. , Intrator, N. and Assaf, Y. (2009), \textit{Free water
                elimination and mapping from diffusion MRI.} Magn. Reson.
                Med., 62: 717-730.
                \href{https://doi.org/10.1002/mrm.22055}{doi:10.1002/mrm.22055}

        \bibitem[Pasternak2014]{Pasternak2014} Pasternak O., Maier-Hein K.,
                Baumgartner C., Shenton M.E., Rathi Y., Westin CF.  (2014)
                \textit{The Estimation of Free-Water Corrected Diffusion
                Tensors.} In: Westin CF., Vilanova A., Burgeth B. (eds)
                Visualization and Processing of Tensors and Higher Order
                Descriptors for Multi-Valued Data. Mathematics and
                Visualization.  Springer, Berlin, Heidelberg
                \href{https://doi.org/10.1007/978-3-642-54301-2\_11}{doi:10.1007/978-3-642-54301-2\_11}

        \bibitem[Kimmel]{Kimmel} Kimmel, Ron. \textit{Geometric Framework in
                Image Processing.} Numerical Geometry of Images: Theory,
                Algorithms, and Applications. N.p.: Springer, 2012. N. pag.
                Print.
                \href{https://doi.org/10.1007/978-0-387-21637-9}{doi:10.1007/978-0-387-21637-9}

\end{thebibliography}


\end{document}
